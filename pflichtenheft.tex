\documentclass[a4paper,10pt]{article}
\usepackage[utf8]{inputenc}
\usepackage{amsmath,amssymb}
\usepackage{enumerate}
\usepackage{ngerman}
\usepackage{graphicx}
\usepackage[left=2.5cm,right=2.5cm,top=2.5cm,bottom=2.5cm]{geometry}
\usepackage[titles]{tocloft}
\usepackage[colorlinks=false]{hyperref}

\title{Pflichtenheft}
\date{}

\author{Usman Ghani Ahmed, Philip Caroli, Maximilian Madlung, Jeremias Mechler, Fabian Neundorf}

\begin{document}

\maketitle
\begin{center}
\huge{Praxis der Softwareentwicklung \\
Gruppe 3 \\[0.5cm]
Entwicklung eines ``Monopoly''-ähnlichen Spiels \\
KIT - Karlsruher Institut für Technologie \\[0.5cm]
WS 2010 / 2011}
\end{center}

\newpage

\tableofcontents

\newpage

\section{Produktübersicht}
Test
\section{Zielbestimmung}
\subsection{Musskriterien}
\subsection{Sollkriterien}
\subsection{Kannkriterien}
\subsection{Abgrenzungskriterien}
\section{Produkteinsatz}
\subsection{Anwendungsbereiche}
\subsection{Zielgruppen}
\subsection{Betriebsbedingungen}
\section{Produktumgebung}
\subsection{Software}
\subsection{Hardware}
\subsection{Orgware}
\subsection{Schnittstellen}
\section{Funktionale Anforderungen}
\section{Produktdaten}
\section{Produktleistungen}
\section{Weitere nichtfunktionale Anforderungen}
\section{Qualitätsanforderungen}
\section{Globale Testfälle und Testszenarien}
\section{Systemmodelle}
\section{Benutzungsoberfläche}
\section{Spezielle Anforderungen an die Entwicklungsumgebung}
\section{Zeit- und Ressourcenplanung}
\section{Ergänzungen}
\section{Glossar}

\end{document}
