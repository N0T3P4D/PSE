\documentclass[a4paper,10pt]{article}
\usepackage[utf8]{inputenc}
\usepackage{amsmath,amssymb}
\usepackage{enumerate}
\usepackage{ngerman}
%\usepackage{graphicx}
\usepackage{ifpdf}
\usepackage[usenames]{color}
\usepackage[left=2.5cm,right=2.5cm,top=2.5cm,bottom=2.5cm]{geometry}
\usepackage[titles]{tocloft}
\usepackage[colorlinks=true,linkcolor=black]{hyperref}
\usepackage[pdftex]{graphicx}
\usepackage{geometry}
%\geometry{top=0cm,bottom=0cm,left=0cm,right=0cm,nohead,nofoot}


\title{Entwurf}
\date{}

\author{Usman Ghani Ahmed \\
Philip Caroli\\
Maximilian Madlung\\ 
Jeremias Mechler\\ 
Fabian Neundorf}

\ifpdf
\DeclareGraphicsExtensions{.pdf,.png}
\else
\DeclareGraphicsExtensions{.eps}
\fi

% Einrückung bei Absätzen
\setlength{\parindent}{0mm}
% Zeilenabstand bei Absätzen
\setlength{\parskip}{2mm}

\begin{document}
 
\vspace{5cm}
\maketitle
\begin{center}
\vspace{3cm}
\huge{Praxis der Softwareentwicklung \\
Gruppe 3 \\[0.5cm]
Entwicklung eines "`Monopoly"'-ähnlichen Spiels \\[0.5cm]
%\includegraphics[height=2cm]{kitlogo_de_rgb}  \\[0.5cm]
WS 2010 / 2011} \\[2cm]
Version 1.1
%\textcolor{red}{! DRAFT !}
\end{center}

\newpage

\tableofcontents

\newpage

\section{KI-Client}

\subsection{Allgemeines}
Feature X ist wegen ... noch nicht voll funktionsfähig
\subsection{Bekannte Fehler / Probleme}
\subsection{Änderungen gegenüber dem Entwurf}
\subsection{Entwurfsentscheidungen}

\section OjimServer
\subsection{Allgemeines}
\subsection {Änderungen gegenüber dem Entwurf}
\begin{itemize}
\item Der Server beinhaltet nun eine Möglichkeit ihn über die Kommandozeile zu starten und zu beenden
\item Der Server bekommt am Anfang die maximale Spielerzahl und die Computerspielerzahl übergeben
\item Die Funktionalität der Methoden aus IServer, IServerTrade und IServerAuction werden wie im Entwurf vorgesehen implementiert, einzelne Abläufe und Folgeaktionen waren im Entwurf aber nicht genau vorgeschrieben.
\item Der Übersichtlichkeit halber wurden einige privaten Methoden eingefügt, beispielsweise
\begin{itemize}
\item private boolean checkAllPlayersReady()
\end{itemize}
\item Es gab ein paar Namens�nderungen im package org.ojim.rmi.server 
\begin{itemize}
\item BufferServer wurde ge�ndert in StartNetOjim
\item ImplBuffer wurde ge�ndert in ImplNetOjim
\end{itemize}
\item Network Klasse wurde weggelassen, da die RMI die Verbingsschicht �berdeckt und die Sockets selber managt
\item Da der Server auch in der Lage sein muss entfernte Methoden in den Clients aufzurufen, sind wir auf Callbacks angewiesen Zu diesem Zweck sind noch folgende Klassen hinzugekommen , die f�r den Server Methoden zur Verf�gung stellen welche aus einer anderen JVM aufgerufen werden k�nnen
\begin{itemize}
\item interface NetClient extends Remote : spezifiziert alle Methoden die �ber das Netzwerk vom Server aufgerufen werden k�nnen 
\item ImplNetClient extends UnicastRemoteObject implements NetOjim,IServer: implementiert die Methoden des Interface NetOjim und leitet die Methodenaufrufe des IServer Objekts �ber RMI weiter , dabei ist es egal ob IServer sich im Netzwerk befindet oder lokal
\item StartNetClient Startet einen neuen Client und holt sich eine Referenz vom Namendienst der im Server gestartet wurde
\end{itemize}
\item Auf Seite des Servers, wurde ein Wrapper erstellt der den Methodenaufruf �ber das Netzwer




\end{itemize}
\end{document}
