\documentclass[a4paper,10pt]{article}
\usepackage[utf8]{inputenc}
\usepackage{amsmath,amssymb}
\usepackage{enumerate}
\usepackage{ngerman}
%\usepackage{graphicx}
\usepackage{ifpdf}
\usepackage[usenames]{color}
\usepackage[left=2.5cm,right=2.5cm,top=2.5cm,bottom=2.5cm]{geometry}
\usepackage[titles]{tocloft}
\usepackage[colorlinks=true,linkcolor=black]{hyperref}
\usepackage[pdftex]{graphicx}
\usepackage{geometry}
%\geometry{top=0cm,bottom=0cm,left=0cm,right=0cm,nohead,nofoot}


\title{Validierung}
\date{}

\author{Usman Ghani Ahmed \\
Philip Caroli\\
Maximilian Madlung\\ 
Jeremias Mechler\\ 
Fabian Neundorf}

\ifpdf
\DeclareGraphicsExtensions{.png,.pdf}
\else
\DeclareGraphicsExtensions{.eps}
\fi

% Einrückung bei Absätzen
\setlength{\parindent}{0mm}
% Zeilenabstand bei Absätzen
\setlength{\parskip}{2mm}

\begin{document}
 
\vspace{5cm}
\maketitle
\begin{center}
\vspace{3cm}
\huge{Praxis der Softwareentwicklung \\
Gruppe 3 \\[0.5cm]
Entwicklung eines "`Monopoly"'-ähnlichen Spiels \\[0.5cm]
%\includegraphics[height=2cm]{kitlogo_de_rgb}  \\[0.5cm]
WS 2010 / 2011} \\[2cm]
%\textcolor{red}{! DRAFT !}
\end{center}

\newpage

\tableofcontents

\newpage

\section{Unit Tests}
\subsection{Testf�lle und Testszenarien aus dem Pflichtenheft}

\section{Integrations- und Belastungstests}

\section{Bugfixing}



\end{document}
