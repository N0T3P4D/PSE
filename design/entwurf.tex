\documentclass[a4paper,10pt]{article}
\usepackage[utf8]{inputenc}
\usepackage{amsmath,amssymb}
\usepackage{enumerate}
\usepackage{ngerman}
\usepackage{graphicx}
\usepackage{ifpdf}
\usepackage[usenames]{color}
\usepackage[left=2.5cm,right=2.5cm,top=2.5cm,bottom=2.5cm]{geometry}
\usepackage[titles]{tocloft}
\usepackage[colorlinks=true,linkcolor=black]{hyperref}

\title{Entwurf}
\date{}

\author{Usman Ghani Ahmed \\
Philip Caroli\\
Maximilian Madlung\\ 
Jeremias Mechler\\ 
Fabian Neundorf}

\ifpdf
\DeclareGraphicsExtensions{.pdf}
\else
\DeclareGraphicsExtensions{.eps}
\fi

% Einrückung bei Absätzen
\setlength{\parindent}{0mm}
% Zeilenabstand bei Absätzen
\setlength{\parskip}{2mm}

\begin{document}
 
\vspace{5cm}
\maketitle
\begin{center}
\vspace{3cm}
\huge{Praxis der Softwareentwicklung \\
Gruppe 3 \\[0.5cm]
Entwicklung eines "`Monopoly"'-ähnlichen Spiels \\[0.5cm]
%\includegraphics[height=2cm]{kitlogo_de_rgb}  \\[0.5cm]
WS 2010 / 2011} \\[2cm]
%\textcolor{red}{! DRAFT !}
\end{center}

\newpage

\tableofcontents

\newpage

\section{Architektur}
\subsection{Eingabe-Validierung durch Server}
\subsubsection{Motivation}
Bei Multiplayer-Spielen besteht die Gefahr, dass Spieler mit illegitimen Mitteln versuchen, sich Vorteile zu verschaffen ("`cheaten"'). Um dies zu verhindern ist angedacht, auf Serverseite alle von den Clients empfangenen Eingaben auf Richtigkeit zu prüfen, sodass an den Server keine falschen Befehle gesendet werden können. Weiteres Ziel ist das Verhindern von Denial-of-Service-Attacken durch Übersenden von unzulässigen oder sinnlosen Befehlen, die dazu geeignet sind, den reibungslosen Spielablauf zu unterbrechen. 
\subsubsection{Klassen}
\begin{itemize}
\item Checker

Diese Klasse implementiert die Interfaces IServer, bla, blub und enthält eine Instanz der Spielregeln und eine Referenz zum eigentlichen Server und zum Spielzustand. Wird eine der in den Interfaces deklarierten Methoden aufgerufen, so wird mithilfe der Spielregeln und des Spielzustandes überprüft, ob diese Eingabe zum jetzigen Zeitpunkt zulässig ist. Ist dies der Fall, wird sie an den Server weitergeleitet, anderenfalls an die Fehlerbehandlungsklasse ErrorHandler übergeben, der über weitere Aktionen entscheidet. Bis auf den Konstruktur, der Referenzen zum Server und zum Spielzustand übernimmt, sind keine öffentlichen Methoden deklariert. Als Attribut wird nur die Spielregeln-Instanz benötigt.
\item ErrorHandler

Je nach unzulässiger Eingabe sind verschiedene Aktionen möglich. Da in unserer Implementierung die Clients ihre Eingaben mit den Spielregeln auf Zulässigkeit überprüfen, kann davon ausgegangen werden, dass Falscheingaben absichtlich herbeigeführt wurden. Entsprechend könnte eine Reaktion daher das sofortige Entfernen eines Spielers aus dem Spiel sein. Um abgestufte Reaktionen zu ermöglichen, könnte man für jeden Spieler ein Punktekonto mit Verstößen führen, wobei ab einem bestimmten Wert Aktionen gegen den Spieler unternommen werden. Öffentliche Methoden sind keine vorhanden, gespeichert wird lediglich eine Referenz auf den Server sowie ggf. eine Liste mit Punkten für jeden Spieler.
\end{itemize}
\subsubsection{Optionales}
Optional könnte noch eine rudimentäre Client-Server-Authentifizierung über einen geheimen Zahlenwert implementiert werden.
\subsection{KI-Client}
\subsection{Interfaces}
\begin{itemize}
\item RatingFunction
Das Interface "`RatingFunction"' dient zur Beschreibung der Bewertungsfunktionen. Der Konstruktur enthält eine Referenz zum Spielzustand, weiterhin gibt es eine öffentlich aufrufbare Bewertungsfunktion, die ein Argument vom Typ X entgegen nimmt und eine Zahl als zurückgibt
\end{itemize}
\subsubsection{Klassen}
\begin{itemize}
\item KiClient
Diese Klasse erbt vom ClientGrundgerüst und überschreibt bzw. implemnentiert Methoden, die KI-spezifisch sind.
\end{itemize}
\subsection{Client Grundgerüst}
\section{Beschreibung der zentralen Abläufe}
\section{Weitere Entwurfdetails}
\section{Abweichungen vom Pflichtenheft}
% besserer Name?
\section{Verantwortlichkeiten und Planung der Implementierungsphase}
\section{Klassendiagramm}
Das Klassendiagramm finden sie als pdf Datei in "`class\_diagram.pdf"'.

\end{document}
